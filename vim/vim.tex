\documentclass{beamer}

\usepackage[utf8]{inputenc}
\usepackage[french]{babel}

\useoutertheme{infolines}
\beamertemplatenavigationsymbolsempty

\AtBeginSection{\frame{\sectionpage}}
\AtBeginSubsection{\frame{\subsectionpage}}
\newtranslation[to=french]{Subsection}{Sous-section}

\title{Vim: éditeur un jour, éditeur toujours}
\author{Guillaume Hélouis}

\begin{document}

\begin{frame}
  \titlepage
\end{frame}

\section{Présentation globale}
\begin{frame} \frametitle{Historique}
  \begin{itemize}
    \item Vi IMproved
    \item Charityware écrit par Bram Moolenaar en 1991
    \item Philosophie UNIX
  \end{itemize}
\end{frame}
\begin{frame} \frametitle{Les modes}
  \begin{itemize}
    \item 6 modes différents
    \item La plupart du temps on reste en mode normal
  \end{itemize}
\end{frame}

\section{Se déplacer}
\begin{frame} \frametitle{Dans le texte}
  \begin{itemize}
    \item Par caractère: h,j,k,l
    \item Par mot: w,e,b
    \item Début/fin de ligne: 0,\$
    \item Début/fin du fichier: gg,G
  \end{itemize}
\end{frame}
\begin{frame} \frametitle{Dans la fenêtre}
  \begin{itemize}
    \item Haut, milieu, bas: H,M,L
  \end{itemize}
\end{frame}

\section{Éditer du texte}
\begin{frame} \frametitle{Insertion}
  \begin{itemize}
    \item Entrer en mode insertion sous le curseur: {\tt i}
    \item En début de ligne: {\tt I}
    \item Au début d'un nouvelle ligne: {\tt o}
    \item Retourner en mode normal: {\tt ESC}
  \end{itemize}
\end{frame}
\begin{frame} \frametitle{Opérations de base}
  \begin{itemize}
    \item Suppression: x
    \item Copier une ligne: yy
    \item Supprimer une ligne: dd
    \item Coller: p
  \end{itemize}
\end{frame}
\subsection{Opérateurs et motions}
\begin{frame} \frametitle{Opérateurs}
  \begin{itemize}
    \item Suppression: d
    \item Copie: y
    \item Préfixer avec n pour répéter n fois
  \end{itemize}
\end{frame}
\begin{frame} \frametitle{Motions}
  \begin{itemize}
    \item Mouvements: w, e, \dots
    \item Délimiteurs: (, \{, [, \textless, ", '
    \begin{itemize}
      \item possible de préfixer avec: a ``arround'', i ``in'', t ``till''
    \end{itemize}
  \end{itemize}
\end{frame}
\begin{frame}[fragile] \frametitle{Mise en pratique}
  \begin{itemize}
    \item \begin{verbatim}[operator][count][motion]\end{verbatim}
  \end{itemize}
  \begin{itemize}
    \item \pause Comment effacer 3 mots? \pause d3w
    \item \pause Comment copier du curseur jusqu'à la fin de la ligne? \pause y\$
    \item \pause Comment copier 50 lignes? \pause 50yy
    \item \pause Comment supprimer tout le texte contenu à l'intérieur de
      quotes? \pause di"
  \end{itemize}
  \pause
  \begin{exampleblock}{Protip}{\tt <ESC>} pour faire un reset\end{exampleblock}
\end{frame}


\section{Les commandes}
\begin{frame} \frametitle{Gestion des fichiers}
  \begin{itemize}
    \item Sauvegarder: {\tt :w[rite]}
    \item Quitter: {\tt :q[uit]}
    \item Quitter sans sauvegarder: {\tt :q!}
    \item Ouvrir un fichier: {\tt :e[dit] <un fichier>}
  \end{itemize}
\end{frame}
\begin{frame}
  \begin{itemize}
    \item Comment sauvegarder et quitter un fichier? \pause {\tt :wq}
  \end{itemize}
  \pause
  \begin{exampleblock}{Protip}{\tt :x (pour exit) fait la même chose!}\end{exampleblock}
\end{frame}
\begin{frame}
  \begin{itemize}
    \item Exécuter une commande shell: {\tt :!<cmd>}
  \end{itemize}
\end{frame}
\begin{frame} \frametitle{Rechercher}
  \begin{itemize}
    \item {\tt /<unMot>}
    \item {\tt n} et {\tt N} pour boucler sur la prochaine/la précédente
      occurrence
  \end{itemize}
  \begin{itemize}
    \item {\tt *} pour aller sur la prochaine occurrence du mot sous le curseur
  \end{itemize}
\end{frame}
\begin{frame} \frametitle{Remplacer}
  \begin{itemize}
    \item Syntaxe sed
    \item Remplacer toutes les occurrences: {\tt :\%s/mot/nouveauMot/g}
  \end{itemize}
\end{frame}
\begin{frame} \frametitle{Aide}
  \begin{itemize}
    \item {\tt :h[elp]}
  \end{itemize}
\end{frame}
\begin{frame} \frametitle{Quiz}
  \begin{itemize}
    \item Comment lister tous les fichiers du répertoire courant sans quitter
      vim? \pause {\tt :!ls} \pause
    \item Comment aller à la prochaine occurrence occurrence du mot sous le
      curseur? \pause {\tt *} \pause
    \item Comment rechercher le mot ``toto''? \pause {\tt /toto} \pause
    \item Comment afficher la documentation de la motion {\tt w}? \pause {\tt
      :help w}
  \end{itemize}
\end{frame}


\section{Configuration}
\begin{frame} \frametitle{Buffer courant}
  \begin{itemize}
    \item Activer une option: {\tt :set <option>}
    \item Désactiver une option: {\tt :set no<option>}
  \end{itemize}
  \pause
  \begin{block}{Exemples}
    \begin{itemize}
      \item {\tt :set number}
      \item {\tt :set nonumber}
    \end{itemize}
  \end{block}
\end{frame}
\begin{frame} \frametitle{.vimrc}
  \begin{itemize}
    \item Options, raccourcis custom, colorscheme \dots
    \item Plugins
  \end{itemize}
  \pause
\end{frame}

\section{Aller plus loin}
\begin{frame}
  \begin{itemize}
    \item {\tt vimtutor}
    \item \href{http://www.openvim.com/}{openvim.com}
    \item \href{https://wiki.archlinux.org/index.php/Vim}{wiki.archlinux.org/index.php/Vim}
    \item \href{https://vim-adventures.com/}{vim-adventures.com}
    \item \href{https://vimgolf.com/}{vimgolf.com}
  \end{itemize}
\end{frame}

\begin{frame}
  \begin{itemize}
    \item Ma config: \href{https://bitbucket.org/ghelouis/config}{bitbucket.org/ghelouis/config}
    \item Les slides: \href{https://github.com/ghelouis/talks}{github.com/ghelouis/talks}
  \end{itemize}
\end{frame}

\end{document}
